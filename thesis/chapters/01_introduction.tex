\chapter{Introduction}

\section{Motivation}

% TODO: Explain why this research is important

Software systems are constantly evolving, requiring regular maintenance and refactoring to ensure code quality, readability, and maintainability. Code refactoring is the process of restructuring existing code without changing its external behavior, aimed at improving the internal structure and design. However, identifying which parts of a codebase require refactoring is a challenging task that typically relies on developer expertise and manual code review.

Recent advances in artificial intelligence, particularly in Large Language Models (LLMs) and diffusion models, have shown remarkable capabilities in understanding and generating code. This thesis explores how these technologies can be combined to automatically identify and localize code segments that would benefit from refactoring.

\section{Problem Statement}

% TODO: Clearly define the problem

The primary problem addressed in this research is the automated identification and localization of code segments requiring refactoring. Current approaches face several challenges:

\begin{itemize}
    \item \textbf{Scalability:} Manual code review is time-consuming and does not scale well for large codebases
    \item \textbf{Consistency:} Different developers may identify different refactoring opportunities
    \item \textbf{Context Understanding:} Traditional static analysis tools lack deep semantic understanding
    \item \textbf{Multi-language Support:} Existing tools often focus on specific programming languages
\end{itemize}

\section{Research Questions}

This thesis addresses the following research questions:

\begin{enumerate}
    \item How can diffusion models be effectively applied to code refactoring localization tasks?
    \item What are the advantages of combining diffusion models with LLMs for code analysis compared to traditional approaches?
    \item How does the proposed approach perform across different programming languages and codebases?
    \item What are the limitations and potential improvements of the proposed method?
\end{enumerate}

\section{Research Objectives}

The main objectives of this research are:

\begin{enumerate}
    \item To develop a novel framework combining diffusion models and LLMs for code refactoring localization
    \item To create a comprehensive dataset for training and evaluating the proposed approach
    \item To implement and evaluate the system across multiple programming languages
    \item To compare the performance with existing baseline methods
    \item To identify limitations and propose future research directions
\end{enumerate}

\section{Contributions}

This thesis makes the following contributions:

\begin{itemize}
    \item \textbf{Novel Architecture:} A new approach combining diffusion models with LLMs for code refactoring localization
    \item \textbf{Comprehensive Evaluation:} Extensive experiments across multiple datasets and programming languages
    \item \textbf{Open-source Implementation:} A publicly available implementation to facilitate future research
    \item \textbf{Empirical Insights:} Analysis of the strengths and limitations of the proposed approach
\end{itemize}

\section{Thesis Organization}

The remainder of this thesis is organized as follows:

\begin{itemize}
    \item \textbf{Chapter 2} provides background on code refactoring, diffusion models, and LLMs
    \item \textbf{Chapter 3} reviews related work in automated refactoring and code analysis
    \item \textbf{Chapter 4} describes the proposed methodology in detail
    \item \textbf{Chapter 5} discusses implementation details and system design
    \item \textbf{Chapter 6} presents the experimental setup and configurations
    \item \textbf{Chapter 7} reports the experimental results
    \item \textbf{Chapter 8} discusses findings, implications, and limitations
    \item \textbf{Chapter 9} concludes the thesis and outlines future work
\end{itemize}
