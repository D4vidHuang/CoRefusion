\chapter{Implementation}

This chapter describes the technical implementation of the proposed system, including software architecture, tools, and libraries used.

\section{System Architecture}

\subsection{Overview}

The implementation follows a modular architecture with clear separation of concerns:

\begin{itemize}
    \item Data processing pipeline
    \item Model training framework
    \item Evaluation suite
    \item Inference engine
\end{itemize}

\subsection{Technology Stack}

\subsubsection{Core Framework}
\begin{itemize}
    \item \textbf{PyTorch}: Deep learning framework
    \item \textbf{Transformers}: Pre-trained model library (Hugging Face)
    \item \textbf{Diffusers}: Diffusion model library
\end{itemize}

\subsubsection{Data Processing}
\begin{itemize}
    \item \textbf{tree-sitter}: Parsing for multiple languages
    \item \textbf{astroid}: Python AST analysis
    \item \textbf{datasets}: Dataset management
\end{itemize}

\subsubsection{Experiment Tracking}
\begin{itemize}
    \item \textbf{Weights \& Biases}: Experiment logging
    \item \textbf{TensorBoard}: Metric visualization
    \item \textbf{Hydra}: Configuration management
\end{itemize}

\section{Data Pipeline}

\subsection{Data Collection}

% TODO: Add data collection flowchart

\subsubsection{Source Selection}
Code samples are collected from:
\begin{itemize}
    \item GitHub repositories with refactoring commits
    \item Open-source projects with documented refactorings
    \item Curated benchmark datasets
\end{itemize}

\subsubsection{Filtering Criteria}
\begin{itemize}
    \item Minimum repository stars: 100
    \item Active maintenance (commits within last year)
    \item Clear license (MIT, Apache, BSD)
    \item Multiple programming languages
\end{itemize}

\subsection{Data Preprocessing}

\subsubsection{Code Parsing}
\begin{lstlisting}[language=Python, caption=Code parsing example]
import tree_sitter

parser = tree_sitter.Parser()
parser.set_language(language)
tree = parser.parse(source_code)
\end{lstlisting}

\subsubsection{Label Generation}
Labels are generated by:
\begin{enumerate}
    \item Analyzing refactoring commits
    \item Extracting modified lines
    \item Annotating refactoring types
\end{enumerate}

\subsubsection{Data Cleaning}
\begin{itemize}
    \item Remove generated code
    \item Filter out trivial changes
    \item Normalize whitespace
    \item Handle encoding issues
\end{itemize}

\subsection{Dataset Splits}

\begin{table}[h]
\centering
\begin{tabular}{lrrr}
\toprule
Split & Files & Lines & Refactorings \\
\midrule
Train & 10,000 & 2M & 50,000 \\
Validation & 2,000 & 400K & 10,000 \\
Test & 2,000 & 400K & 10,000 \\
\bottomrule
\end{tabular}
\caption{Dataset statistics (example)}
\label{tab:dataset_stats}
\end{table}

\section{Model Implementation}

\subsection{Code Encoder}

\begin{lstlisting}[language=Python, caption=Code encoder implementation]
class CodeEncoder(nn.Module):
    def __init__(self, model_name):
        super().__init__()
        self.encoder = AutoModel.from_pretrained(model_name)
        self.projection = nn.Linear(hidden_size, embed_dim)
    
    def forward(self, input_ids, attention_mask):
        outputs = self.encoder(
            input_ids=input_ids,
            attention_mask=attention_mask
        )
        return self.projection(outputs.last_hidden_state)
\end{lstlisting}

\subsection{Diffusion Module}

\begin{lstlisting}[language=Python, caption=Diffusion model core]
class DiffusionModel(nn.Module):
    def __init__(self, num_timesteps, hidden_dim):
        super().__init__()
        self.num_timesteps = num_timesteps
        self.denoising_network = DenosingNetwork(hidden_dim)
    
    def forward(self, x, t, condition):
        return self.denoising_network(x, t, condition)
\end{lstlisting}

\subsection{Training Loop}

\begin{lstlisting}[language=Python, caption=Training loop structure]
def train_epoch(model, dataloader, optimizer):
    model.train()
    total_loss = 0
    
    for batch in tqdm(dataloader):
        optimizer.zero_grad()
        
        # Forward pass
        loss = model.compute_loss(batch)
        
        # Backward pass
        loss.backward()
        torch.nn.utils.clip_grad_norm_(
            model.parameters(), max_norm=1.0
        )
        optimizer.step()
        
        total_loss += loss.item()
    
    return total_loss / len(dataloader)
\end{lstlisting}

\section{Evaluation Implementation}

\subsection{Metrics Computation}

\begin{lstlisting}[language=Python, caption=Metrics evaluation]
def compute_metrics(predictions, labels):
    precision = precision_score(labels, predictions)
    recall = recall_score(labels, predictions)
    f1 = f1_score(labels, predictions)
    
    return {
        'precision': precision,
        'recall': recall,
        'f1': f1
    }
\end{lstlisting}

\subsection{Baseline Implementations}

Baseline methods implemented for comparison:
\begin{itemize}
    \item Rule-based detector
    \item Static analysis baseline
    \item Traditional ML classifier
    \item Standard LLM without diffusion
\end{itemize}

\section{Optimization Techniques}

\subsection{Memory Optimization}

\begin{itemize}
    \item Gradient checkpointing
    \item Mixed precision training (FP16)
    \item Efficient attention mechanisms
    \item Batch size tuning
\end{itemize}

\subsection{Computational Efficiency}

\begin{itemize}
    \item Multi-GPU training (DataParallel/DistributedDataParallel)
    \item Gradient accumulation
    \item Efficient data loading
    \item Model quantization for inference
\end{itemize}

\section{Reproducibility}

\subsection{Random Seed Management}

\begin{lstlisting}[language=Python, caption=Reproducibility setup]
import random
import numpy as np
import torch

def set_seed(seed=42):
    random.seed(seed)
    np.random.seed(seed)
    torch.manual_seed(seed)
    torch.cuda.manual_seed_all(seed)
    torch.backends.cudnn.deterministic = True
\end{lstlisting}

\subsection{Configuration Management}

All experiments are configured using YAML files managed by Hydra, ensuring reproducibility.

\subsection{Version Control}

\begin{itemize}
    \item Code versioning with Git
    \item Model checkpoint versioning
    \item Dataset versioning
    \item Dependency management with requirements.txt
\end{itemize}

\section{Deployment Considerations}

\subsection{Inference API}

A REST API for model inference:
\begin{itemize}
    \item FastAPI framework
    \item Model serving with TorchServe
    \item Containerization with Docker
\end{itemize}

\subsection{Scalability}

\begin{itemize}
    \item Batch inference support
    \item Caching mechanisms
    \item Load balancing
\end{itemize}

\section{Development Tools}

\subsection{Code Quality}

\begin{itemize}
    \item \textbf{black}: Code formatting
    \item \textbf{flake8}: Style checking
    \item \textbf{pylint}: Static analysis
    \item \textbf{mypy}: Type checking
\end{itemize}

\subsection{Testing}

\begin{itemize}
    \item \textbf{pytest}: Unit testing framework
    \item \textbf{pytest-cov}: Coverage reporting
    \item Continuous integration with GitHub Actions
\end{itemize}

\section{Summary}

This chapter detailed the implementation of the proposed system, including the technology stack, data pipeline, model implementation, and development practices. The next chapter describes the experimental setup.
