\chapter{Discussion}

This chapter interprets the experimental results, discusses implications, identifies limitations, and addresses threats to validity.

\section{Interpretation of Results}

\subsection{Why Diffusion Models Help}

The superior performance of the diffusion-based approach can be attributed to:

\begin{itemize}
    \item \textbf{Uncertainty modeling:} Diffusion models naturally capture uncertainty in refactoring decisions
    \item \textbf{Gradual refinement:} The iterative denoising process allows progressive refinement of predictions
    \item \textbf{Context integration:} The conditioning mechanism effectively incorporates code context
    \item \textbf{Distribution matching:} Learning the distribution of refactoring locations rather than direct classification
\end{itemize}

\subsection{Component Contributions}

\subsubsection{Pre-trained Code Encoder}

The large improvement from pre-training demonstrates the importance of:
\begin{itemize}
    \item Large-scale code understanding
    \item Transfer learning from related tasks
    \item Rich semantic representations
\end{itemize}

\subsubsection{Structural Features}

Structural features capture important aspects:
\begin{itemize}
    \item Code organization
    \item Complexity metrics
    \item Dependency patterns
\end{itemize}

\subsection{Cross-language Performance}

The variation in performance across languages reflects:
\begin{itemize}
    \item Dataset size differences
    \item Language-specific idioms
    \item Varying refactoring practices
    \item Parser quality differences
\end{itemize}

\section{Comparison with Baselines}

\subsection{Traditional Methods}

Traditional rule-based and metric-based approaches are limited by:
\begin{itemize}
    \item Reliance on predefined patterns
    \item Lack of context understanding
    \item High false positive rates
    \item Language-specific implementations
\end{itemize}

\subsection{Machine Learning Baselines}

Classical ML and standard deep learning methods show:
\begin{itemize}
    \item Better generalization than rule-based methods
    \item Limited semantic understanding
    \item Difficulty with long-range dependencies
    \item Need for extensive feature engineering
\end{itemize}

\subsection{LLM-based Baselines}

Standard LLM fine-tuning provides strong performance but:
\begin{itemize}
    \item Lacks explicit uncertainty modeling
    \item May overfit to training patterns
    \item Limited interpretability
\end{itemize}

The diffusion-based approach addresses these limitations through probabilistic modeling.

\section{Implications}

\subsection{For Software Engineering Practice}

\subsubsection{Automated Code Review}

The system can assist in:
\begin{itemize}
    \item Prioritizing code review efforts
    \item Identifying technical debt
    \item Supporting continuous refactoring
    \item Maintaining code quality
\end{itemize}

\subsubsection{Developer Support}

Potential applications include:
\begin{itemize}
    \item IDE integration for real-time suggestions
    \item Pre-commit hooks
    \item CI/CD pipeline integration
    \item Educational tools for learning refactoring
\end{itemize}

\subsection{For Research}

\subsubsection{Diffusion Models for Code}

This work demonstrates that diffusion models can be effectively applied to code analysis tasks, opening possibilities for:
\begin{itemize}
    \item Other code generation and analysis tasks
    \item Improved uncertainty quantification
    \item Better handling of ambiguous cases
\end{itemize}

\subsubsection{Multi-modal Code Understanding}

The combination of different representations (tokens, AST, metrics) suggests benefits from:
\begin{itemize}
    \item Multi-modal learning approaches
    \item Hierarchical representations
    \item Fusion of complementary information sources
\end{itemize}

\section{Limitations}

\subsection{Dataset Limitations}

\subsubsection{Coverage}

\begin{itemize}
    \item Limited to open-source projects
    \item Potential bias toward popular languages
    \item May not reflect proprietary codebase patterns
    \item Historical bias in refactoring practices
\end{itemize}

\subsubsection{Labeling Quality}

\begin{itemize}
    \item Automated label extraction may introduce noise
    \item Different developers have different refactoring preferences
    \item Temporal context not always captured
\end{itemize}

\subsection{Model Limitations}

\subsubsection{Scalability}

\begin{itemize}
    \item Computational cost higher than simpler baselines
    \item Memory requirements limit context length
    \item Inference time may be prohibitive for very large codebases
\end{itemize}

\subsubsection{Generalization}

\begin{itemize}
    \item Performance degrades on rare refactoring types
    \item Limited cross-domain transfer
    \item Requires substantial training data
\end{itemize}

\subsubsection{Interpretability}

\begin{itemize}
    \item Diffusion process is less interpretable than rule-based methods
    \item Difficult to explain specific predictions
    \item Limited actionable feedback for developers
\end{itemize}

\subsection{Evaluation Limitations}

\subsubsection{Metrics}

\begin{itemize}
    \item Standard metrics may not fully capture usefulness
    \item No direct measure of developer productivity impact
    \item Difficulty measuring opportunity cost of missed refactorings
\end{itemize}

\subsubsection{User Study}

\begin{itemize}
    \item Limited scale of user evaluation
    \item Potential bias in participant selection
    \item Controlled setting may not reflect real usage
\end{itemize}

\section{Threats to Validity}

\subsection{Internal Validity}

Potential threats include:
\begin{itemize}
    \item Hyperparameter selection bias
    \item Implementation bugs
    \item Non-deterministic training
    \item Data leakage
\end{itemize}

\textbf{Mitigation strategies:}
\begin{itemize}
    \item Systematic hyperparameter search
    \item Extensive testing and code review
    \item Multiple runs with different seeds
    \item Careful data split validation
\end{itemize}

\subsection{External Validity}

Generalization concerns:
\begin{itemize}
    \item Limited to studied programming languages
    \item Focus on specific refactoring types
    \item Open-source project bias
    \item Temporal generalization not tested
\end{itemize}

\textbf{Mitigation strategies:}
\begin{itemize}
    \item Evaluation on diverse datasets
    \item Multiple language evaluation
    \item Benchmark dataset inclusion
\end{itemize}

\subsection{Construct Validity}

Measurement concerns:
\begin{itemize}
    \item Metrics may not capture true refactoring quality
    \item Ground truth ambiguity
    \item Evaluation granularity choices
\end{itemize}

\textbf{Mitigation strategies:}
\begin{itemize}
    \item Multiple complementary metrics
    \item Qualitative analysis
    \item User study validation
\end{itemize}

\subsection{Conclusion Validity}

Statistical concerns:
\begin{itemize}
    \item Multiple comparisons
    \item Sample size adequacy
    \item Assumption violations
\end{itemize}

\textbf{Mitigation strategies:}
\begin{itemize}
    \item Bonferroni correction
    \item Statistical power analysis
    \item Non-parametric tests where appropriate
\end{itemize}

\section{Unexpected Findings}

\subsection{Emergent Behaviors}

Several unexpected observations:
\begin{itemize}
    \item Model sometimes identifies valid refactorings not in ground truth
    \item Attention patterns reveal novel code smell indicators
    \item Cross-language transfer better than expected for syntactically similar languages
\end{itemize}

\subsection{Failure Mode Analysis}

Detailed analysis reveals:
\begin{itemize}
    \item Specific patterns that confuse the model
    \item Context length limitations more impactful than expected
    \item Importance of training data diversity
\end{itemize}

\section{Lessons Learned}

\subsection{Technical Insights}

\begin{itemize}
    \item Importance of careful conditioning in diffusion models
    \item Value of multi-level representations
    \item Need for domain-specific adaptations
    \item Benefits of probabilistic approaches for ambiguous tasks
\end{itemize}

\subsection{Methodological Insights}

\begin{itemize}
    \item Comprehensive evaluation beyond standard metrics
    \item Importance of qualitative analysis
    \item Value of ablation studies
    \item Need for diverse baseline comparisons
\end{itemize}

\section{Summary}

This chapter discussed the implications of results, identified limitations, and addressed threats to validity. Key takeaways include:

\begin{itemize}
    \item Diffusion models offer advantages for uncertain, context-dependent tasks
    \item The approach shows promise for practical software engineering applications
    \item Several limitations suggest directions for future improvement
    \item Rigorous evaluation strengthens confidence in findings
\end{itemize}

The next chapter concludes the thesis and outlines future research directions.
